\documentclass[]{article}
\usepackage{lmodern}
\usepackage{amssymb,amsmath}
\usepackage{ifxetex,ifluatex}
\usepackage{fixltx2e} % provides \textsubscript
\ifnum 0\ifxetex 1\fi\ifluatex 1\fi=0 % if pdftex
  \usepackage[T1]{fontenc}
  \usepackage[utf8]{inputenc}
\else % if luatex or xelatex
  \ifxetex
    \usepackage{mathspec}
  \else
    \usepackage{fontspec}
  \fi
  \defaultfontfeatures{Ligatures=TeX,Scale=MatchLowercase}
\fi
% use upquote if available, for straight quotes in verbatim environments
\IfFileExists{upquote.sty}{\usepackage{upquote}}{}
% use microtype if available
\IfFileExists{microtype.sty}{%
\usepackage{microtype}
\UseMicrotypeSet[protrusion]{basicmath} % disable protrusion for tt fonts
}{}
\usepackage[margin=1in]{geometry}
\usepackage{hyperref}
\hypersetup{unicode=true,
            pdftitle={Chapter 2},
            pdfborder={0 0 0},
            breaklinks=true}
\urlstyle{same}  % don't use monospace font for urls
\usepackage{graphicx,grffile}
\makeatletter
\def\maxwidth{\ifdim\Gin@nat@width>\linewidth\linewidth\else\Gin@nat@width\fi}
\def\maxheight{\ifdim\Gin@nat@height>\textheight\textheight\else\Gin@nat@height\fi}
\makeatother
% Scale images if necessary, so that they will not overflow the page
% margins by default, and it is still possible to overwrite the defaults
% using explicit options in \includegraphics[width, height, ...]{}
\setkeys{Gin}{width=\maxwidth,height=\maxheight,keepaspectratio}
\IfFileExists{parskip.sty}{%
\usepackage{parskip}
}{% else
\setlength{\parindent}{0pt}
\setlength{\parskip}{6pt plus 2pt minus 1pt}
}
\setlength{\emergencystretch}{3em}  % prevent overfull lines
\providecommand{\tightlist}{%
  \setlength{\itemsep}{0pt}\setlength{\parskip}{0pt}}
\setcounter{secnumdepth}{0}
% Redefines (sub)paragraphs to behave more like sections
\ifx\paragraph\undefined\else
\let\oldparagraph\paragraph
\renewcommand{\paragraph}[1]{\oldparagraph{#1}\mbox{}}
\fi
\ifx\subparagraph\undefined\else
\let\oldsubparagraph\subparagraph
\renewcommand{\subparagraph}[1]{\oldsubparagraph{#1}\mbox{}}
\fi

%%% Use protect on footnotes to avoid problems with footnotes in titles
\let\rmarkdownfootnote\footnote%
\def\footnote{\protect\rmarkdownfootnote}

%%% Change title format to be more compact
\usepackage{titling}

% Create subtitle command for use in maketitle
\newcommand{\subtitle}[1]{
  \posttitle{
    \begin{center}\large#1\end{center}
    }
}

\setlength{\droptitle}{-2em}

  \title{Chapter 2}
    \pretitle{\vspace{\droptitle}\centering\huge}
  \posttitle{\par}
    \author{}
    \preauthor{}\postauthor{}
    \date{}
    \predate{}\postdate{}
  

\begin{document}
\maketitle

\section{Hard question 1}\label{hard-question-1}

\emph{Suppose there are two species of Species Bear. Both are equally
common in the wild and live in the same places. They look exactly alike
and eat the same food, and there is yet no genetic assay capable of
telling them apart. They differ however in their family sizes. Species A
gives birth to twins 10\% of the time, otherwise birthing a single
infant. Species B births twins 20\% of the time, otherwise birthing
singleton infants. Assume these numbers are known with certainty, from
many years of field research.}

\emph{Now suppose you are managing a captive Species Breeding program.
You have a new female panda of unknown species, and she has just given
birth to twins. What is the probability that her next birth will also be
twins?}

The important information about the two species.

Species A:

\begin{itemize}
\tightlist
\item
  10\% Twins, \(\textrm{P}(\textrm{Twin} \mid \textrm{Species A})\)
\item
  90\% Single infant
\end{itemize}

Species B:

\begin{itemize}
\tightlist
\item
  20\% Twins, \(\textrm{P}(\textrm{Twin} \mid \textrm{Species B})\)
\item
  80 \% Single infant
\end{itemize}

The problem is divided up using total probaility:

\[  \textrm{P}(\textrm{Twin} \mid \textrm{Previous twin})  =  \textrm{P}(\textrm{Twin} \mid \textrm{Species A}) \times \textrm{P}(\textrm{Species A} \mid \textrm{Previous twin}) + \textrm{P}(\textrm{Twin} \mid \textrm{Species B}) \times \textrm{P}(\textrm{Species A} \mid \textrm{Previous twin}) \]

Two of the probabilites are given directly in the excersise. Inserting
these into the expression gives
\[  \textrm{P}(\textrm{Twin} \mid \textrm{Previous twin})  =  0.1 \times\textrm{P}(\textrm{Species A} \mid \textrm{Previous twin}) + 0.2 \times \textrm{P}(\textrm{Species A} \mid \textrm{Previous twin}) \]

Calculating the missing probabilites is done using bayes rule. First for
Species A
\[ \textrm{P}(\textrm{Species A} \mid \textrm{Previous twin})    = \frac{ \textrm{P}( \textrm{Previous twin} \mid \textrm{Species A} ) \times \textrm{P}(\textrm{Species A})}{\textrm{P}( \textrm{Previous twin})} = \frac{0.1 \times 0.5}{\textrm{P}(\textrm{Previous twin})}\]

To complete the calculation the probability of a twin is needed. Total
probaility can be used
\[ \textrm{P}(\textrm{Previous twin}) = \textrm{P}(\textrm{Previous twin} \mid \textrm{Species A}) \times \textrm{P}(\textrm{Species A}) +  \textrm{P}(\textrm{Previous twin} \mid \textrm{Species B}) \times \textrm{P}(\textrm{Species B}) = 0.1 \times 0.5 + 0.2 \times 0.5 = 0.15 \]

Inserting this back into the bayes rule applied on the probability the
panda being from Species A completes that calculation
\[ \textrm{P}(\textrm{Species A} \mid \textrm{Previous twin})    =  \frac{0.1 \times 0.5}{0.15} = 0.33\]

The same calculation is repeated for Species B
\[ \textrm{P}(\textrm{Species B} \mid \textrm{Previous twin})    =  \frac{ \textrm{P}( \textrm{Previous twin} \mid \textrm{Species B} ) \times \textrm{P}(\textrm{Species B})}{\textrm{P}( \textrm{Previous twin})} = \frac{0.2 \times 0.5}{0.15} = 0.66 \]

Finally the two probabilites calcutated using Bayes rule can go back
into the original expression
\[  \textrm{P}(\textrm{Twin} \mid \textrm{Previous twin})  =  0.1 \times\textrm{P}(\textrm{Species A} \mid \textrm{Previous twin}) + 0.2 \times \textrm{P}(\textrm{Species A} \mid \textrm{Previous twin}) = 0.1  \times  0.33  + 0.2  \times 0.66  =  0.165 = 16.5\%\]

16.5\% makes sence. The probability is more than just taking the average
probability of having twins from the two species. SInce the previous
litter was twins, it increased the probability of the panda being from
species B and therefore the probability of the panda giving birth to
twins.

\section{Hard Question 2}\label{hard-question-2}

\emph{Recall all the facts from the problem above. Now compute the
probability that the panda we have is from species A, assuming we have
observed only the first birth and that it was twins.}

This was calculated in question 1

\[ \textrm{P}(\textrm{Species A} \mid \textrm{Previous twin})    = \frac{ \textrm{P}( \textrm{Previous twin} \mid \textrm{Species A} ) \times \textrm{P}(\textrm{Species A})}{\textrm{P}( \textrm{Previous twin})} = \frac{0.1 \times 0.5 }{0.15} =0.33 = 33\%\]

\section{Hard Question 3}\label{hard-question-3}

Continuing on from the previous problem, suppose the same panda mother
has a second birth and that it is not twins, but a singleton infant.
Compute the posterior probability that this panda is species A.

The question expressed in symbols becomes
\[ \textrm{P}(\textrm{Species A} \mid ( \textrm{Second singelton} \cap  \textrm{First twin} ))   \]

\subsection{Aproach 1}\label{aproach-1}

This can be solved using bayes
\[ \textrm{P}(\textrm{Species A} \mid ( \textrm{Second singelton} \cap  \textrm{First twin} ))   = \frac{\textrm{P}( ( \textrm{Second singelton} \cap  \textrm{First twin} ) \mid \textrm{Species A} ) \times \textrm{P}(   \textrm{Species A} )}{ \textrm{P}( \textrm{Second singelton} \cap  \textrm{First twin} )}  \]
\[   \frac{\textrm{P}( ( \textrm{Second singelton} \cap  \textrm{First twin} ) \mid \textrm{Species A} ) \times \textrm{P}(   \textrm{Species A} )}{ \textrm{P}( \textrm{Second singelton} \cap  \textrm{First twin} )} = \frac{ \textrm{P}(\textrm{Second singelton}  \mid \textrm{Species A}) \times \textrm{P}(  \textrm{First twin}  \mid \textrm{Species A})  ) \times \textrm{P}(   \textrm{Species A} ) }{ \textrm{P}( \textrm{Second singelton} \cap  \textrm{First twin} ) }\]
Using that the two events: the second being a singelton given the
species is A i and the first being a twin given the species is A, are
independent. We can calulate the probaiblity useing multiplication.

\[   \frac{ \textrm{P}(\textrm{Second singelton}  \mid \textrm{Species A}) \times \textrm{P}(  \textrm{First twin}  \mid \textrm{Species A})  ) \times \textrm{P}(   \textrm{Species A} ) }{ \textrm{P}( \textrm{Second singelton} \cap  \textrm{First twin} ) } = \frac{0.9 \times 0.1 \times 0.5}{\textrm{P}( \textrm{Second singelton} \cap  \textrm{First twin} )} \]

To adress the denominator we use total pobability.
\[  {\textrm{P}( \textrm{Second singelton} \cap  \textrm{First twin} )} =  {\textrm{P}( (\textrm{Second singelton} \cap  \textrm{First twin} ) \mid  \textrm{Species A})} \times  \textrm{P}(   \textrm{Species A} ) +  {\textrm{P}( (\textrm{Second singelton} \cap  \textrm{First twin} ) \mid  \textrm{Species B})} \times \textrm{P}(   \textrm{Species B} )\]
Using that the conditional probabilites are independent the probabilites
can be calulated using multiplication.
\[  {\textrm{P}( \textrm{Second singelton} \cap  \textrm{First twin} )} = 0.9 \times 0.1 \times 0.5 +  0.8 \times 0.2 \times 0.5 = 0.125 \]
Inserting this

\[ \textrm{P}(\textrm{Species A} \mid ( \textrm{Second singelton} \cap  \textrm{First twin} )) = \frac{0.9 \times 0.1 \times 0.5}{0.125} = 0.36 = 36\%\]
\#\# Aproach 2 Appraoch 1 was allot of work. Stating with what we know
from question 2 it can be done faseter.

\textbf{The prior is now updated and with it so are the probabilites.
Keeping the notation simple \(\textrm{P}(\textrm{Species A}\) now takes
a new value without an updated notation.}

\[ \textrm{P}(\textrm{Species A} \mid  \textrm{Second singelton})  =   \frac{\textrm{P}( \textrm{Second singelton}\mid\textrm{Species A} )\textrm{P}(\textrm{Species A})}{\textrm{P}(\textrm{Second singeltion}) }  = \frac{ 0.9 \times  0.33}{\textrm{P}(\textrm{Second singeltion}) }\]
\[\textrm{P}(\textrm{Second singeltion}) = \textrm{P}(\textrm{Second singeltion} \mid \textrm{Species A}) \times \textrm{P}(\textrm{Species A})  + \textrm{P}(\textrm{Second singeltion} \mid \textrm{Species B}) \times \textrm{P}(\textrm{Species A}) = 0.9 \times 0.33 + 0.8 \times 0.66 = 0.825\]

\[ \textrm{P}(\textrm{Species A} \mid  \textrm{Second singelton})  = \frac{ 0.9 \times  0.33}{\textrm{P}(\textrm{Second singeltion}) } =  \frac{ 0.9 \times  0.33}{ 0.825 } = 0.36 = 36\% \]

\section{Hard Question 4}\label{hard-question-4}

A common boast of Bayesian statisticians is that Bayesian inference
makes it easy to use all of the data, even if the data are of different
types.

So suppose now that a veterinarian comes along who has a new genetic
test that she claims can identify the species of our mother panda. But
the test, like all tests, is imperfect. This is the information you have
about the test: - The probability it correctly identifies a species A
panda is 0.8. - The probability it correctly identifies a species B
panda is 0.65. The vet administers the test to your Species And tells
you that the test is positive for species A. First ignore your previous
information from the births and compute the posterior probability that
your panda is species A. Then redo your calculation, now using the birth
data as well.

Ignoring the adjustemnts to the priors we use bayes
\[\textrm{P}(\textrm{Species A} \mid  \textrm{Tested as A})  = \frac{\textrm{P}(  \textrm{Tested as A}\mid  \textrm{Species A} ) \times \textrm{P}(\textrm{Species A})}{ \textrm{P}(\textrm{Tested as A}) } = \frac{0.8 \times 0.5}{\textrm{P}(\textrm{Tested as A})}\]
\[\textrm{P}(\textrm{Tested as A}) =  \textrm{P}(\textrm{Tested as A} \mid \textrm{Species A}) \times \textrm{P}(\textrm{Species A}) +  \textrm{P}(\textrm{Tested as A} \mid \textrm{Species B}) \times \textrm{P}(\textrm{Species B}) = 0.8 \times 0.5 + 0.35 \times 0.5 = 0.575\]

\[\textrm{P}(\textrm{Species A} \mid  \textrm{Tested as A})  = \frac{0.8 \times 0.5}{\textrm{P}(\textrm{Tested as A})} = \frac{0.8 \times 0.5}{ 0.575 } = 0.696 = 69.6\%\]

Including the updated priors due to the testing gives anoter result. As
in question 3 approach 2 this means the probabilites are updated wihout
updating the notation.

\[\textrm{P}(\textrm{Species A} \mid  \textrm{Tested as A})  = \frac{\textrm{P}(  \textrm{Tested as A}\mid  \textrm{Species A} ) \times \textrm{P}(\textrm{Species A})}{ \textrm{P}(\textrm{Tested as A}) } = \frac{0.8 \times 0.36}{\textrm{P}(\textrm{Tested as A})}\]

\[\textrm{P}(\textrm{Tested as A}) =  \textrm{P}(\textrm{Tested as A} \mid \textrm{Species A}) \times \textrm{P}(\textrm{Species A}) +  \textrm{P}(\textrm{Tested as A} \mid \textrm{Species B}) \times \textrm{P}(\textrm{Species B}) = 0.8 \times 0.36 + 0.35 \times 0.56 = 0.484\]

\[\textrm{P}(\textrm{Species A} \mid  \textrm{Tested as A})  = \frac{0.8 \times 0.36}{\textrm{P}(\textrm{Tested as A})} = \frac{0.8 \times 0.36}{  0.484}  =  0.595 = 59.5\% \]


\end{document}
